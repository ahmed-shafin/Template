\chapter{Various}

\section{Intervals}
	\kactlimport{IntervalContainer.h}
	\kactlimport{IntervalCover.h}
	\kactlimport{ConstantIntervals.h}

\section{Misc. algorithms}
    \kactlimport{TernaryDouble.h}
	\kactlimport{TernarySearch.h}
	\kactlimport{LIS.h}
	\kactlimport{FastKnapsack.h}
    \kactlimport{Hackenbush.h}

\section{Dynamic programming}
	\kactlimport{KnuthDP.h}
	\kactlimport{DivideAndConquerDP.h}

\section{Optimization tricks}
	\subsection{Bit hacks}
		\begin{itemize}
            \item \verb@__builtin_popcount(unsigned int x)@ - Counts the number of set bits
            \item \verb@__builtin_clz(unsigned int x)@ - Counts the leading zeroes in an integer
            \item \verb@__builtin_ctz(unsigned int x)@ - Counts the trailing zeroes in an integer
			\item \verb@x & -x@ is the least bit in \texttt{x}.
			\item \verb@for (int x = m; x; ) { --x &= m; ... }@ loops over all subset masks of \texttt{m} (except \texttt{m} itself).
			\item \verb@c = x&-x, r = x+c; (((r^x) >> 2)/c) | r@ is the next number after \texttt{x} with the same number of bits set.
			\item \verb@rep(b,0,K) rep(i,0,(1 << K))@ \\ \verb@  if (i & 1 << b) D[i] += D[i^(1 << b)];@ computes all sums of subsets.
		\end{itemize}
